This was the first version of the assignment. After the start of the project, the assignment has changed.

\section*{Project description}

It is well known that many students come to the University to visit lectures because it provides the opportunity to meet peers and to have discussions also about less academic topics. The meeting place at the University is partly replaced by a virtual meeting place. Students report about their daily activities, opinions, problems using social media. At this moment social media play a limited role in the academic process of teaching and learning. But social media offer great opportunities, especially around the development of MOOCS (Massive Open Online Courses).The basic idea of MOOCS is that students remote in place and time follow the lectures, they can use social media as academic communication media, where students discuss about their study activities, put forward their questions and get help and support from peers. The question is how to find the best matching peers especially peers outside TUDelft. Inspecting all the profiles via Facebook is too time consuming. Supporting matching tools are needed. At this moment there many sites finding the best matching partner, our interest is finding the best matching study partners. A possible option is to extract features from student profiles via FaceBook , take care of preferences and expert knowledge and train classifiers or other matching technologies. Students involved in the peer matching project are supposed to communicate via social media. But at the end of the project there will be a written report. The TUDelft supervisor Leon Rothkrantz is involved in supervising students for many years. Currently he participates in an European project FETCH aimed at the development of new didactical models for distant learning. This gives the current Peer matching project an European dimension and offers cooperation with students from other European Universities.

\section*{Company description}

The TUDelft is well known, so no additional info is provided. At this moment there is a lot of interest in the development of MOOCS. The Board of the University (Anka Mulder) supports the developments for many reasons. In the framework of The European Life Long Learning Program TUDelft is a partner in the FETCH project (Future Education and Training in Computing). More than 55 European Universities are involved in this project. There is a special Work Package on the use of social media in distant learning. Prof. drs. dr. L.J.M. Rothkrantz is the coordinator of this project. There are possibilities for students to take part in this project and present their work on Educational Conferences organised by the project.

\section*{Auxiliary information}

The company FeedbackFruits will be represented by Ewoud de Kok (ewoud@feedbackfruits.com +31630496917). The company is cofounded by TUDelft graduates (computer science). The have experience in developing software products according to well known standards. last years many students did an internship at the company He will have the role of case holder or client of the project. An interesting aspect is to study how online educational platforms will change the content and didactical aspects. It is now a fact that many students are online almost the whole day. Regular courses at the University are too static, boring and most important not interactive. It proves that even workgroups and interaction during classes are less effective. It is difficult to measure the effectivity/efficiency and to make effective changes in the way of teaching.Online courses have a broad spectrum of didactical forms. It ranges from video lectures, courses with quizzes and tutorials at regular times, courses modelled as games, and a lot of gimmicks which gives learning a lot of funs Online courses have a lot of social activities (not only knowledge transfer), students will be stimulated and enabled to help each other during hangouts and group meetings via Skype-like interfaces. This enables students to give feedback and discussions about the content, with or without the lecturer. The educational challenge is to integrate for example the use of social media in online courses. This will have a huge impact on the way Universities will offer educational and training material.

Company: TUDelft, Fetch\\
TU Delft coach: Leon Rothkrantz\\
TU Delft coach email address: L.J.M.Rothkrantz@tudelft.nl\\
Company contact: Leon Rothkrantz\\
Company contact email: l.j.m.rothkrantz@tudelft.nl