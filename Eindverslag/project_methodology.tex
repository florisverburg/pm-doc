\chapter{Project Methodology}


\section{Process Strategy}


\subsection{Software development method}
We have used scrum as our software development method.
Scrum is based on the agile software development framework which can be used to manage software projects.
We have chosen scrum because of its flexible approach to software development.
We did not have the time to research all of the aspects of the system.
Scrum gave us the possibility to react to changes of requirements, but also to react to new insights given by our increasing understanding of the system as time progresses.
We have defined the different roles associated with scrum (e.g. the product owner, development team, scrum master) as you can see below.
\\\\
\noindent\textbf{Stakeholders}\\
The following table lists the different group members, their roles and a short description.
This description describes the role and responsibilities of the group member.\\
\begin{tabular}{|l|l|p{5cm}|}
\hline
Name & Roles & Description\\
\hline
Freek van Tienen & Lead Development & Responsible for implementation phase.\\
\hline
Floris Verburg & Scrum master and Quality Assurance & Responsible for the Scrum process, the quality of the code and test coverage.\\
\hline
Marijn Goedegebure & Project Manager and Reports & Manages the project and ensures that the requirements are met.\\
\hline
Leon Rothkrantz & Client, Group Mentor and Supervisor & Supervises the project, aides the group with difficulties and has issued the assignment.\\
\hline
Dragos Datcu &  Group Mentor and Supervisor & Supervises the project and aides the group with difficulties.\\
\hline
\end{tabular}

\section{Planning}
We have set up a global planning that splits up the project in phases.
Each of these phases has deliverables attached to them.
We will be mentioning our goals for each week.
Other important dates are also added to the planning.

\subsubsection{Phase 1, Set up and research}
During the first phase we will be researching the requirements of the system and which requirements are essential for the application and what requirements are not.
We will also be looking into the set up of the project for our implementation phase.\\

\noindent{Week 1: 21-04-2014 - 25-04-2014}\\
Drafts of Plan of Approach and literature study, setup of the development environment.\\
\\
\noindent\emph{Week 2: 28-04-2014 - 02-05-2014}\\
Finished Plan of Approach, literature study, first system design.

\subsubsection{Phase 2, Implementation}
During the second phase we will be developing the application.
During this time we will be using Scrum to support our development.
During week 3 to 8 we will be developing the application.
During this time we won't have big deliverables, but we will be using the weekly meetings with our supervisors to measure our progress.\\

\noindent\emph{Week 3: 05-05-2014 - 09-05-2014}\\

\noindent\emph{Week 4: 12-05-2014 - 16-05-2014}\\

\noindent\emph{Week 5: 19-05-2014 - 23-05-2014}\\

\noindent\emph{Week 6: 26-05-2014 - 30-05-2014}\\
Start of the acceptance testing.
\\\\
\noindent\emph{Week 7: 02-06-2014 - 06-06-2014}\\
End of the acceptance testing, first draft of the final report and send code to SIG for review.

\subsubsection{Phase 3, Final report and presentation}

\noindent\emph{Week 8: 09-06-2014 - 13-06-2014}\\
Final development week, final draft of the final report.
\\\\
\noindent\emph{Week 9: 16-06-2014 - 20-06-2014}\\
Send code for final review to SIG, send final report to supervisors for review, creation of the presentation.
\\\\
\noindent\emph{Week 10: 23-06-2014 - 27-06-2014}\\
Presentation

\section{Tools}
We have used several tools to support our development. They helped us to maintain the quality of the code and improve the maintainability. 

\begin{itemize}
\item IntelliJ IDEA, is our Integrated Development Environment (IDE) which will facilitate the development and testing of our code.
IntelliJ will be integrated with the Play framework which will be helpful during development.
\item GitHub, is our on-line code and documentation repository.
It provides for a version control system, an issue tracker and code review possibilities.
We will be using it to store all our code and documentation's code (LaTeX).
\item Cloudbees
Cloudbees will be used to run our test environment, our continuous integration and the release environment.
It uses Jenkins for the continuous integration.
\item Play framework
The play framework gives us the possibility to easily create a web application using Java.
\item Findbugs is a plug in for Cloudbees and IntelliJ that gives us the possibility to let our java code be checked for small bugs using static analysis.
\item JaCoCo is a plug in for Cloudbees and IntelliJ that provides us with data analysis about our code coverage.
\item Checkstyle is a plug in for Cloudbees and IntelliJ that checks the code for coding standards.
This makes it ideal to enforce the coding standard for our project.
\end{itemize}
