\chapter{Project Methodology}
In this chapter we will describe the project methodology that we have applied to this project.
We will first divide the project into three phases.
For each phase we will specify the goal of this phase, the planning, the process strategy, the planned deliverables and the project roles.
Lastly, we define the tools that we have used during the project.
We have chosen for this particular form of documentation because in this way we can easily desribe the unique aspects of each phase.

\section{Global planning}


\section{Phase 1}
\subsection{Goal}
The first phase aims to expand the knowledge of the group about the problem and it's requirements.
Secondly, it also gives time for the team members to analyse the problem for difficult subjects and to spend time finding a fitting solution.
Thirdly it gives the team members the time to set up the project infrastructure and to define the project methodology.
Lastly, the remaining time should be spend making a first start with the system design.

\subsection{Planning}
We planned phase 1 to be two weeks long.
With the first week having a focus on setting up the project and doing research to the problem and requirements.
The second week was more about defining the project methodology, analysing and researching the possible problems that we could encounter and making a first system design.
The second week was also used to finish up the research report.
\subsection{Planned deliverables}
At the end of phase 1 we wanted to have the following deliverables done:
\begin{itemize}
\item Research report
\item Plan of approach
\item First system design
\end{itemize}

\subsection{Process strategy}
During this phase we choose to not have a strict task distribution.
Many of the tasks at hand needed group thinking and group opinions.
For example, the intrepretation of the problem description is something that can be thought out by a single person, but needs to be communicated about toward the other team members.
If not communicated enough, the resulting problem description may not be the same as how the team interpreted the client's description.
Things that needed to be done were documented in a spreadsheet.
In this way, every team member had insight into the things that still needed to be done.

\subsection{Project roles}
As said above, there was not a strict task destribution.
Following the advise of our group mentor, every week a different team member was the designated chairman and another team member was the designated secretary.
The chairman of the week had the responsibility to preside the weekly meetings and to maintain an overview of the tasks that were at hand.
The secretary had the responsibility of making minutes of both the weekly meetings and the meetings with the group mentors and client.
Next to these two roles, there were no other defined roles.

\subsection{Phase reflection}
%Nog te doen: schrijven over het verloop van deze phase

\section{Phase 2}
\subsection{Goal}
The purpose of the second phase was to provide a time frame in which both a basic system design can be made and the system design could be implemented.
During the second phase the roles were more strict since we wanted to follow a structured software development method.
This software development method is the framework known as scrum.
We will go into depth about this in the process strategy section.

\subsection{Planning}
We planned the second phase to be six weeks long.
The first week would be used to finish up the system design.
The other 5 weeks will be used to implement this system design.
During the final two weeks we looked into the user tests and the acceptance tests.
During this time there we also sent our source code to the Software Improvement Group for them to check.
The feedback that we got back will be 

\subsection{Planned deliverables}
During phase two we wanted to have the following deliverables done:
\begin{itemize}
\item Basic system design
\item User tests
\item Send the source code to SIG for the first review
\end{itemize}

\subsection{Process strategy}
During the implementation phase we used a known software development framework.
This development framework is known as scrum.
Scrum is based on the agile software development framework which can be used to manage software projects.
We have chosen scrum because of its flexible approach to software development.
We did not have the time to research all of the aspects of the system.
Scrum gave us the possibility to react to changes of requirements, but also to react to new insights given by our increasing understanding of the system as time progresses.
We have defined the different roles associated with scrum (e.g. the product owner, development team, scrum master) as you can see below.

\subsection{Project roles}
The following table lists the different group members, their roles and a short description.
This description describes the role and responsibilities of the group member.\\
\begin{tabular}{|l|l|p{5cm}|}
\hline
Name & Roles & Description\\
\hline
Freek van Tienen & Lead Development & Responsible for implementation phase.\\
\hline
Floris Verburg & Scrum master and Quality Assurance & Responsible for the Scrum process, the quality of the code and test coverage.\\
\hline
Marijn Goedegebure & Project Manager and Reports & Manages the project and ensures that the requirements are met.\\
\hline
Leon Rothkrantz & Client, Group Mentor and Supervisor & Supervises the project, aides the group with difficulties and has issued the assignment.\\
\hline
Dragos Datcu &  Group Mentor and Supervisor & Supervises the project and aides the group with difficulties.\\
\hline
\end{tabular}
\subsection{Phase reflection}
%Nog te doen: schrijven over het verloop van deze phase

\section{Phase 3}
\subsection{Goal}
The goal of the third and last phase is to provide for time to finish up the project.
During this phase time has been reserved for finishing up the implementation, the final report and for the user tests.

\subsection{Planning}

\subsection{Planned deliverables}

\subsection{Process strategy}

\subsection{Project roles}

\subsection{Phase reflection}


\section{Planning}
We have set up a global planning that splits up the project in phases.
Each of these phases has deliverables attached to them.
We will be mentioning our goals for each week.
Other important dates are also added to the planning.

\subsubsection{Phase 3, Final report and presentation}

\noindent\emph{Week 8: 09-06-2014 - 13-06-2014}\\
Final development week, final draft of the final report.
\\\\
\noindent\emph{Week 9: 16-06-2014 - 20-06-2014}\\
Send code for final review to SIG, send final report to supervisors for review, creation of the presentation.
\\\\
\noindent\emph{Week 10: 23-06-2014 - 27-06-2014}\\
Presentation

\section{Tools}
We have used several tools to support our development.
They helped us to maintain the quality of the code and improve the maintainability.
Some of these tools also helped with structuring the implementation phase.

\begin{itemize}
\item IntelliJ IDEA, is our Integrated Development Environment (IDE) which will facilitate the development and testing of our code.
IntelliJ will be integrated with the Play framework which will be helpful during development.
\item GitHub, is our on-line code and documentation repository.
It provides for a version control system, an issue tracker and code review possibilities.
We will be using it to store all our code and documentation's code (LaTeX).
\item Cloudbees
Cloudbees will be used to run our test environment, our continuous integration and the release environment.
It uses Jenkins for the continuous integration.
\item Play framework
The play framework gives us the possibility to easily create a web application using Java.
\item Findbugs is a plug in for Cloudbees and IntelliJ that gives us the possibility to let our java code be checked for small bugs using static analysis.
\item JaCoCo is a plug in for Cloudbees and IntelliJ that provides us with data analysis about our code coverage.
\item Checkstyle is a plug in for Cloudbees and IntelliJ that checks the code for coding standards.
This makes it ideal to enforce the coding standard for our project.
\end{itemize}
