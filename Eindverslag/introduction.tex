\chapter{Introduction}

\section{Problem Description}
It is well known that many students come to the University to visit lectures because it provides the opportunity to meet peers and to have discussions also about less academic topics. 
The meeting place at the University is partly replaced by a virtual meeting place. 
Students report about their daily activities, opinions, problems using social media. 

At this moment social media plays a limited role in the academic process of teaching and learning. 
But social media offers great opportunities, especially around the development of MOOCS (Massive Open Online Courses). 

The basic idea of MOOCS is that students, remote in place and time, follow the lectures. 
They can use social media as academic communication media, where students discuss about their study activities, put forward their questions and get help and support from fellow students. 

The question is how to find the best matching fellow students, especially students outside TUDelft. 
Inspecting the profiles from all these students via Facebook is too time consuming, therefore are supporting matching tools needed. 

At this moment there many sites finding the best matching partner, our interest is finding the best matching study partners. 
A possible option is to extract features from student profiles via Facebook or LinkedIn, train classifiers or other matching technologies and take care of preferences and expert knowledge.

\section{Target}
In this project we focus mainly on students, but teachers are also important for accomplishing a good result.
We need to match the students based on their skills, but the teachers are the ones that determine the demands for the assignments and what the optimal level of average skills in a group is.

We focus on creating optimal practical groups for students.
So it is also very important to test the application with these people.

\section{Outline}
We will identify the assignment based on the requirements we determined in Chapter~\ref{sec:requirements}. 
In this chapter, we also give an impression on how the application will look like by means of graphical user interface (GUI) mockups.
Then, in Chapter~\ref{sec:methodology} the strategy and methodology of the process will be explained by going into more detail on our approach, the project planning and tools used throughout the project.
Next, in Chapter~\ref{sec:design_implementation} the implementation details and design decisions will be explained.
We will discuss why we chose for certain implementations of for example the algorithm or the system design.
In Chapter~\ref{sec:testing} we will discuss our way of testing. 
We differentiate the unit testing and the user testing.
Finally, we conclude the report in Chapter~\ref{sec:conclusion} by evaluating the results and providing suggestions for future work.

After the conclusion a Bibliography and several Appendices can be found. 
These appendices contain intermediate products and more detailed descriptions that are referred to in the text in order to improve the readability of the main report.
