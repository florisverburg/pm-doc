\chapter{Introduction}

\section{Problem Description}
It is well known that many students come to the University to visit lectures because it provides the opportunity to meet peers and to have discussions also about less academic topics. 
The meeting place at the University is partly replaced by a virtual meeting place. 
Students report about their daily activities, opinions, problems using social media. 
\\\\
At this moment social media play a limited role in the academic process of teaching and learning. 
But social media offer great opportunities, especially around the development of MOOCS (Massive Open Online Courses). 
\\\\
The basic idea of MOOCS is that students remote in place and time follow the lectures, they can use social media as academic communication media, where students discuss about their study activities, put forward their questions and get help and support from peers. 
\\\\
The question is how to find the best matching peers especially peers outside TUDelft. 
Inspecting all the profiles via Facebook is too time consuming. 
Supporting matching tools are needed. 
\\\\
At this moment there many sites finding the best matching partner, our interest is finding the best matching study partners. 
A possible option is to extract features from student profiles via Facebook, take care of preferences and expert knowledge and train classifiers or other matching technologies.

\section{Project Assignment}
MOOCs around the world are growing in size and thus the practical assignments need to be easy to manage.
The growth also means that it is more difficult for each participant to find a possible practical partner due to the large amount of people.
It also becomes really difficult to find the correct partner or group members when you don't see each other during courses and haven't even met most of the people from your course.
\\\\
Most of the time participants of MOOCs are searching group members on social media like Facebook or Twitter.
They just try to look through some of the participants names in the course, search for them on-line through social media and check if they are a capable partner or group member.
This approach takes a lot of time and most of the time there are too much MOOC participants to go trough all of them.
\\\\
Next to the fact that this process is very time consuming, this process also often leads to a sub optimal solution.
This is due to the problem that people often look for partners of group members which share the same interests.
While this often makes it easy to work together, this will not guarantee that you have a good functioning group and enough knowledge inside the group.
\\\\
We would like to improve upon this process by creating an application that can be used to create practical groups without the participants being required to go search intensively trough all the participants.
Participants should be able to easily access the application, fill in some relevant information about themselves and then be recommended possible practical partners or group members.
\\\\
The interaction with the system has some difficulties.
If we want to get some information from third parties, we are required to use the information in a way this third party delivers the data.
There can be difficulties processing this data.
\\\\
The recommendation of possible group members can be quite difficult too.
One of the problems we are facing is the question: What user information is useful to determine whether two people can be good practical partners?

\section{Target}
In this project, we focus mainly on students, but teachers are also important for accomplishing a good result.
We need to match the students based on their skills, but the teachers are the ones that determine the demands for the assignments and what the optimal level of average skills in a group is.
\\\\
We focus on creating optimal practical groups for students.
So it is also very important to test the application with these people.

\section{Outline}
We will identify the assignment based on the requirements we determined in chapter 2. 
In this chapter, we also give an impression on how the application will look like by means of graphical user interface (GUI) mockups.
\\\\
Then, in chapter 3 the strategy and methodology of the process will be explained by going into more detail on our approach, the project planning and tools used throughout the project.
\\\\
Next, in chapter 4 the implementation details and design decisions will be explained.
We will discuss why we chose for certain implementations of for example the algorithm or the system design.
\\\\
In chapter 5 we will discuss our way of testing. 
We differentiate the unit testing and the user testing.
\\\\
Finally, we conclude the report in chapter 6 by evaluating the results and providing suggestions for future work.
\\\\
After the conclusion a Bibliography and several Appendices can be found. 
These appendices contain intermediate products and more detailed descriptions that are referred to in the text in order to improve the readability of the main report.
