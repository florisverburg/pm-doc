\chapter*{Summary}
\setheader{Summary}

Massive open online courses (MOOCs) all around the world are growing in size. 
Therefore, the practical assignments need to be easy to manage.
The growth also means that it is more difficult for each participant to find a possible practical partner due to the large amount of people.
The difficulty to find the correct partner or group members when the students do not see each other during courses and have not even met most of the people from the course is also affected.

For this Bachelor Project we have improved upon this process by creating an application that can be used to create practical groups without the participants being required to go search intensively trough all the participants.
Participants are able to easily access the application, fill in some relevant information about themselves and then get recommendations of possible practical partners or group members.

We have divided the project in three phases:
\begin{itemize}
\item Research and set-up (phase 1)
\item Design and implementation (phase 2)
\item Testing, documentation and presentation (phase 3)
\end{itemize}
Each of these phases describes the things that need to be done during that phase.

During the implementation phase we used a known software development framework.
This development framework is known as scrum.
Scrum is based on the agile software development framework which can be used to 
manage software projects.

We use the Play framework for the developing of the application. 
The Play framework gives us the possibility to easily create a web application using Java.
This frameworks follows the Model View Controller (MVC) architectural pattern applied to the Web architecture.

For our matching algorithm we used our custom designed algorithm.
This algorithm uses the skills defined by the user and the teacher to find the best matching with the preferences of the teacher.


Creating code with good quality, maintainability and extendability of the system was important during the project.
We achieved this using several tools, like Findbugs and Checkstyle.

Another important aspect of the project was testing the code.
We have performed different kind of tests to ensure good functioning code.

During the implementation phase, we already started with testing. Because the system was not completed during this phase, only parts of the system could be tested.
We performed different kind of unit tests to provide a basic code coverage.
We also used continuous integration, which executed the tests ever time we pushed our code to git.
Besides the unit tests we have performed integration tests to ensure the connection between the different components.

The testing of the entire system is done after the implementation phase has been completed. This is partly done by checking if the final product matches the requirements set. It is also done by letting the target audience use the system and document the findings.

A working beta implementation of the complete system has been delivered as a proof of concept and most of the initially set requirements were fulfilled.
