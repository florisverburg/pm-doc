\documentclass[]{article}
\usepackage{a4wide}
\usepackage{xcolor}
\usepackage[numbers]{natbib}
\usepackage[colorlinks,linkcolor=black,urlcolor=blue,citecolor=black]{hyperref}

\newcommand{\TODO}[1]{{\color{red}\textbf{TODO: #1}}}
\newcommand{\reqr}[1]{{\noindent\emph{#1:}}}
\renewcommand*\contentsname{Table of Contents}

\title{Peer Matching Research}
\author{Marijn Goedegebure \and
	Floris Verburg \and
	Freek van Tienen}
\date{}

\begin{document}
\maketitle

\begin{abstract}
This document will explain the design process of the Peer Matching system made for the MOOCs, Massive Online Open Courses.
\TODO{Some more text}
\end{abstract}

\newpage

\tableofcontents

\newpage

\section{Introduction}
It is well known that many students come to the university to visit lectures because it provides the opportunity to meet peers and to have discussions also about less academic topics.
At this moment social media plays a limited role in the academic process of teaching and learning. However, social media offers great opportunities, especially around the development of MOOCS (Massive Open Online Courses).

The basic idea of MOOCS is that students remote in place and time follow the lectures, they can use social media as an academic communication media, where students discuss about their study activities, put forward their questions and get help and support from peers. 

At this moment there many sites finding the best matching partner, our goal is finding the best matching study partners.

\section{Problem definition}
More and more people from all over the world with different backgrounds are participating in the MOOCs.
During most of the MOOCs practical assignments need to be made with a partner or in a group.
But finding the correct partner or group members can become really difficult when you don't see each other during courses and haven't even met most of the people from you course.

Most of the time people are searching partners or group members on social media like Facebook or Twitter.
They just try to look through some of the participants names in the course, search for them online trough social media and check if they are a capable partner or group member.
This approach takes a lot of time and most of the time there are too much MOOC participants to go trough all of them.

These approaches often doesn't lead to the "most optimal" solution, but more to a group or set of partners which share the same interest and are able to work together.
Next to the fact that you want a socially stable group or practicum partner you also want to combine as much different experiences as possible.
For example if both practicum partners a good at programming but also both don't like, or aren't good at writing documents and papers.
Then you would be better of by having one person who is good at programming and a person who is better at writing documents.

\TODO{Finish problem definition}

\section{Introduction to requirements}
The following sections will describe what requirements are set for the product.
This section will define four global requirements who cover the entire project.
These requirements are then split up in the next two sections, the first being the requirements for the basic system and the second being the requirements for the peer matching algorithm.
We choose to split these requirements up, because of the literature study required to further specify the requirements.
In these sections, the global requirements are split up in smaller requirements.
It is easier to verify these smaller requirements for their completion and these requirements are more transformable to functions of the system.
We also assigned these requirements into their appropiate MoSCoW \cite{highsmith2001agile} method category.

\subsection{Actors of the system}
\TODO{Define the actors of the system and provide a brief introduction}
\subsubsection{MOOC participant}
\subsubsection{MOOC teacher/MOOC administrator} 
\subsubsection{Groups of participants}
This actor consist of two or more MOOC participants who accepted each other to their group.schetsen

\subsection{Global Requirements}
To provide an overview of what the software must be capable of, we define four global requirements for the system. By defining these four global requirements we provide a foundation of what the system must be able to do.
\begin{itemize}
\item The system will be using user information to try to find the best practical partner.
The user must be able to enter his personal data so it can be used by the algorithm.
This user information is further defined in what kind of information is important (a must have) and what information is less important (i.e. a should/could have) in the appropiate section.
\item The system must be designed to accomodate the usage for both MOOC administrators and participants.
Each of these actors needs to be able to perform different actions then the other.
For example the participants must be able to register to the system and must be able to join a MOOC.
The section that will further specify the usage of the system requirements, will define what actions are a must and what actions are less important.
\item Next to the actions different actors can perform using the system, the system must also be defined in other terms.
These terms are about for example about the accesability of the application and the usage of a database.
The requirements specified in the section about the kind of system will set demands which will help to select the appropiate framework and kind of application.
\item The system is centered around an algorithm used to recommend a practical partner to a participant.
There are requirements that are set for this algorithm concerning the input it must be able to handle and the output it must give back.
\end{itemize}

\section{Requirements basic system}
This section will mention the different requirements that involve the basic system which the actors interact with. The section starts with mentioning the possible types of user information, followed up by what actions are possible with the system, lastly the kind of systems are limited by the requirements listed.
\subsection{User information}
This subsection will provide information about what data is interesting for our peer matching algorithm and how this data is acquired.
Some data is necessary to collect. 
Other data would be nice to have to give a better result, but is not essential for basic matching.
We distinguish the essential data and the not essential data.

\reqr{Must have}
\begin{itemize}
\item Users must have a viewable profile that can be filled with personal information.
\item Users must be able to enter the following registration information during registration: Full name, Email address, Password and Language.
\end{itemize}

\reqr{Should have}
\begin{itemize}
\item Users should be able to enter the following basic user information: Country, Age, Short description, Skills, Education and Experiences.
\end{itemize}

\reqr{Could have}
\begin{itemize}
\item Users could be able to enter the following extended user information: Preferred group Role\TODO{More information}
\item Users could be able to enter a predefined questionnaires.
\end{itemize}

\reqr{Would have}
\begin{itemize}
\item \TODO{Add would haves to the peer matching requirements}
\end{itemize}

\subsection{Usage of the system}

\subsubsection{Actor 1: MOOC participant}

\reqr{Must have}
\begin{itemize}
\item The a potential participant (anyone who visits the site) must be able to view the different MOOCs without registering or logging in.
\item The participant must be able to inform him/herself about the usage of the application without registering or logging in.
\item The participant must be able to register to the application.
\item The participant must be able to log in to the application.
\item The participant must be able to register to the MOOCs which they were enlisted for (known to the administrator).
\item The participant must be able to view a possible practical partner.
\item The participant must be able to communicate with the possible practical partner to verify the matching.
\item The participant must be able to accept or decline the practical partner into his practical group.
\item The participant must be able to communicate within the group.
\end{itemize}
\TODO{Add must haves to the peer matching requirements}

\reqr{Should have}
\begin{itemize}
\item \TODO{Add should haves to the peer matching requirements}
\end{itemize}

\reqr{Could have}
\begin{itemize}
\item \TODO{Add could haves to the peer matching requirements}
\end{itemize}

\reqr{Would have}
\begin{itemize}
\item \TODO{Add would haves to the peer matching requirements}
\end{itemize}


\subsubsection{Actor 2: MOOC teacher/MOOC administrator}

\reqr{Must have}
\begin{itemize}
\item The administrator must be able to create a MOOC and add this to his account.
\item The administrator must be able to open and close the particpant registration for a MOOC.
\item The administrator must be able to manage the list of participants and groups of participants.
\item The administrator must be able to view what groups have finished what assignments.
\end{itemize}

\reqr{Should have}
\begin{itemize}
\item The administrator should be able to grade the group for its practical.
\item The administrator should be able to set deadlines for each practical.
\item The administrator should be able to easily communicate with every participant.
\item The administrator should be able to communicate with the groups.
\end{itemize}

\reqr{Could have}
\begin{itemize}
\item The administrator could be able to provide for his own criteria for each of his MOOC.
\item The administrator could be able to add different assignments to the application.
\item The administrator could be able to grade the different assignments.
\item The administrator could be able to provide feedback for the different assignments.
\end{itemize}

\reqr{Would have}
\begin{itemize}
\item \TODO{Add would haves to the peer matching requirements}
\end{itemize}


\subsubsection{Actor 3: Groups of participants}
This actor consist of two or more MOOC participants who accepted each other to their group.

\reqr{Must have}
\begin{itemize}
\item The groups must be able to communicate with each other using the application.
\end{itemize}

\reqr{Should have}
\begin{itemize}
\item \TODO{Add should haves to the peer matching requirements}
\end{itemize}

\reqr{Could have}
\begin{itemize}
\item The participants could be able to share files to other group members using this application.
\end{itemize}

\reqr{Would have}
\begin{itemize}
\item The participants would be able upload their source code to the application for verification.
\item \TODO{Add would haves to the peer matching requirements}
\end{itemize}

\subsection{Kind of system}

\reqr{Must have}
\begin{itemize}
\item The application must be easily accessible to both participants and MOOC organizers.
\item The system must provide account management functionality.
\item The system must use a database to store the different data put in by the user.
\end{itemize}

\reqr{Should have}
\begin{itemize}
\item \TODO{Add should haves to the peer matching requirements}
\end{itemize}

\reqr{Could have}
\begin{itemize}
\item The system could be able to be combined with practical systems already used by MOOC organizers.
\end{itemize}

\reqr{Would have}
\begin{itemize}
\item \TODO{Add would haves to the peer matching requirements}
\end{itemize}

\section{Peer matching requirements and literature study}
The peer matching algorithm can be limited and pushed in the correct direction by the requirements, but leaves room for choices to be made.
To make the correct choices a literature study is required.
This section will first list the requirements set by the client and will continue with a discussion about different possibilities to choose from.
It will discuss a range of possible algorithms that can be applied to the problem and what their respective advantages and disadvantages are.
This section will end with a conclusion about the algorithm we will be using.

\subsection{Peer matching}
This subsection will provide a first idea for the peer matching algorithm and what requirements are made for this algorithm.
This algorithm has all of the information of each user as input and calculates values from this information that can be used to compare and group users.
Using these values, the algorithm can calculate if two users would make a good team.
The input is defined by the previous subsection and the output will be defined in this subsection.
\TODO{Finish Peer matching requirements}

\reqr{Must have}
\begin{itemize}
\item The algorithm must be able to use a list of users as input.
\item The algorithm must be able to return one or more suggestions of practical partners to the user given a list of rejected users.
\end{itemize}

\reqr{Should have}
\begin{itemize}
\item The algorithm uses basic user provided data (e.g. education, preferences) to build up a profile for a MOOC participant.
\item The algorithm can calculate whether two people will be good practical partners using the basic user provided data.
\item The algorithm can return a possible practical partners using the basic user provided data.
\end{itemize}

\reqr{Could have}
\begin{itemize}
\item The algorithm uses advanced user provided data (e.g. asked questions, programming tests) to build up an extended profile for a MOOC participant.
\item The algorithm can calculate whether two people will be good practical partners using the advanced user provided data.
\item The algorithm can return a possible practical partner using the basic user provided data.
\end{itemize}

\reqr{Would have}
\begin{itemize}
\item The algorithm would have to be able to use information gathered from rejections to improve the matching.
\end{itemize}

\subsection{Literature study}
\cite{cmu}
\TODO{Add different algorithms and a comparison}

\section{Chosen implementations for requirements}
Some requirements limit the possible choices for a part of the system, but still leaves some questions unanswered.
This section will provide an answer to these questions.\\
\TODO{Continue working on this section}\\
\TODO{Add requirements that leave room for different implementations (e.g. peer matching algorithm, what method is used)}

\section{Conclusion}

\newpage
\bibliographystyle{plainnat}
\bibliography{references}

\end{document}