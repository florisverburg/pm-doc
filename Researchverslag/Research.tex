\documentclass[]{article}
\usepackage{a4wide}
\usepackage{xcolor}
\usepackage[numbers]{natbib}
\usepackage[colorlinks,linkcolor=black,urlcolor=blue,citecolor=black]{hyperref}

\newcommand{\TODO}[1]{{\color{red}\textbf{TODO: #1}}}
\newcommand{\reqr}[1]{{\noindent\emph{#1:}}}
\renewcommand*\contentsname{Table of Contents}

\title{Peer Matching Research}
\author{Marijn Goedegebure \and
	Floris Verburg \and
	Freek van Tienen}
\date{}

\begin{document}
\maketitle

\begin{abstract}
This document will explain the design process of the Peer Matching system made for the MOOCs, Massive Online Open Courses.
\TODO{Some more text}
\end{abstract}

\TODO{
Nieuwe structuur\\
- Problem description\\
- Requirements(af te leiden uit het probleem)\\
-- Global requirements\\
-- User information requirements\\
-- Algorithm requirements\\
- Design choises(af te leiden uit de requirements / literatuur)\\
-- Web application\\
-- Framework\\
-- Algorithm\\}

\newpage

\tableofcontents

\newpage

\section{Preface}
This document is the research report that has been written for the Peer Matching assignment for the Bachelor Project course. This course is the last project based course of the Bachelor of Science at the TU Delft in the Netherlands. The assignment was issued by Dr. L.J.M. Rothkrantz of the TU Delft.
The report provides a summary of the requirements that are set for the system and will provide for a literature study about different possible algorithms to solve the peer matching problem presented in the problem definition.

\section{Introduction}
Many students go to an university each year to follow the predefined bachelor and/or master programs, but not everyone has the time and intent to follow an entire program.
There is a demand from all sorts of people to be able to follow a course while their main occupation isn't that of a student.
For example: a working parent could be very interested in such a course.
That is why universities introduced MOOCs.

The basic idea of MOOCs is that students remote in place and time follow the lectures, can participate in a practical assignment and can make an exam.
If they score sufficient, they are rewarded with a certificate that states their succes in the course.

The concept of the MOOCs is a very new concept that still has many areas to improve upon.
One of these areas is the area of practical assignments.
Currently, for a Computer Science MOOC at the TU Delft, multiple practical assignments are required.
The participants are required to form groups to make these assignments.
After the announcement the participants either use mail, Facebook or  Twitter to contact other people in order to form the groups.

We would like to improve upon this concept and try to provide a more fitting approach.

\subsection{Problem definition}
In this subsection we will continue on the case provided by the introduction.
We will pinpoint the points we would like to improve upon.\\

MOOCs around the world are growing in size and thus the practical assignments need to be easy to manage.
The growth also means that it is more difficult for each participant to find a possible practical partner due to the large amount of people.
It also becomes really difficult to find the correct partner or group members when you don't see each other during courses and haven't even met most of the people from your course.

Most of the time participants of MOOCs are searching group members on social media like Facebook or Twitter.
They just try to look through some of the participants names in the course, search for them on-line through social media and check if they are a capable partner or group member.
This approach takes a lot of time and most of the time there are too much MOOC participants to go trough all of them.

Next to the fact that this process is very time consuming for every participant, it also leads to a sub optimal solution.
This sub optimal solution will be focussed on a group or set of partners which share the same interest and this doesn't say anything if they would form a good group.

We would like to improve upon this process by creating an application that can be used to create practical groups without the participants being required to go search intensively.
Participants should be able to easily access the application, fill in some relevant information about themselves and then be recommended possible group members.

The interaction with the system is a pretty straight forward application, but the recommendation of possible group members is more difficult.
One of the questions we are trying to answer is: What user information is useful to determine whether two people can be good practical partners?
We will also need to figure out a good way to recommend a practical partner, but still give the possibility to reject such a request.

These questions and difficulties will be answered in this report.

\section{Global system design}
In this section we will be describing a first concept of the system.
In this concept we will describe the different actions, that different actors will need to be able to perform.
Using this concept we can later define the requirements.\\

The first thing that has to be done is for the teacher to create a assignment in the application.
It must be possible for the teacher to open and close this assignment. 
While creating, the teacher has to define some properties of the assignment.
When the assignment is created successfully, a link is generated that can be sent to the students.

When the students open the link, they have a choice between logging in or registering.
For logging in and registering, the student could use LinkedIn, so a lot of data is gathered automatically.
After filling in the basic information, the student must provide information about his personality.
This can be done by means of an questionnaire.

After logging in, the assignment can be added to the profile of the student.
After adding the assignment to the profile, some recommendations are provided.
The student can form groups by means of this recommendations.

\section{Introduction to requirements}
The following sections will describe what requirements are set for the product.
This section will define four global requirements who cover the entire project.
These requirements are then split up in the next two sections, the first being the requirements for the basic system and the second being the requirements for the peer matching algorithm.
We choose to split these requirements up, because of the literature study required to further specify the requirements.
In these sections, the global requirements are split up in smaller requirements.
It is easier to verify these smaller requirements for their completion and these requirements are more transformable to functions of the system.
We also assigned these requirements into their appropiate MoSCoW \cite{highsmith2001agile} method category.

\subsection{Actors of the system}
In this subsection we will clarify the different actors used in the system and throughout this report.
\subsubsection{MOOC participant}
Due to the nature of the MOOCs, there are no requirements set for the participants.
Every person with an interest in the given course can enlist for the MOOC and should thus be able to register with our system.
\subsubsection{MOOC teacher/MOOC administrator} 
The MOOC is organised by a professor of an university.
This professor will need to register with the system and create a MOOC.
\subsubsection{Groups of participants}
This actor consist of two or more MOOC participants who accepted each other to their group.
These groups can be of variable size, set by the MOOC organiser.

\subsection{Global Requirements}
To provide an overview of what the software must be capable of, we define four global requirements for the system.
By defining these four global requirements we provide a foundation of what the system must be able to do.
\begin{itemize}
\item The system will be using user information to try to find the best practical partner.
The user must be able to enter his personal data so it can be used by the algorithm.
This user information is further defined in what kind of information is important (a must have) and what information is less important (i.e. a should/could have) in the appropiate section.
\item The system must be designed to accomodate the usage for both MOOC administrators and participants.
Each of these actors needs to be able to perform different actions then the other.
For example the participants must be able to register to the system and must be able to join a MOOC.
The section that will further specify the usage of the system requirements, will define what actions are a must and what actions are less important.
\item Next to the actions different actors can perform using the system, the system must also be defined in other terms.
These terms are about for example about the accesability of the application and the usage of a database.
The requirements specified in the section about the kind of system will set demands which will help to select the appropiate framework and kind of application.
\item The system is centered around an algorithm used to recommend a practical partner to a participant.
There are requirements that are set for this algorithm concerning the input it must be able to handle and the output it must give back.
\end{itemize}

\section{Requirements basic system}
This section will mention the different requirements that involve the basic system which the actors interact with.
The section starts with mentioning the possible types of user information, followed up by what actions are possible with the system, lastly the kind of systems are limited by the requirements listed.

\subsection{User information}
This subsection will provide information about what data is interesting for our peer matching algorithm and how this data is acquired.
Some data is necessary to collect. 
Other data would be nice to have to give a better result, but is not essential for basic matching.
We distinguish the essential data and the not essential data.

\reqr{Must have}
\begin{itemize}
\item Users must have a viewable profile that can be filled with personal information.
\item Users must be able to enter the following registration information during registration: Full name, Email address, Password and Language.
\end{itemize}

\reqr{Should have}
\begin{itemize}
\item Users should be able to enter the following basic user information: Country, Age, Short description, Skills, Education, a Picture and Personality Traits.
\end{itemize}

\reqr{Could have}
\begin{itemize}
\item Users could be able to use a personality test to determine their personality traits.
\item Users could be able to fill in their personality traits.
\item Users could be able to enter the following extended user information: Preferred group Role\TODO{More information}
\item Users could be able to enter a predefined questionnaires.
\end{itemize}

\reqr{Would have}
\begin{itemize}
\item \TODO{Add would haves to the peer matching requirements}
\end{itemize}

\subsection{Usage of the system}

\subsubsection{Actor 1: MOOC participant}

\reqr{Must have}
\begin{itemize}
\item The a potential participant (anyone who visits the site) must be able to view the different MOOCs without registering or logging in.
\item The participant must be able to inform him/herself about the usage of the application without registering or logging in.
\item The participant must be able to register to the application.
\item The participant must be able to log in to the application.
\item The participant must be able to register to the MOOCs which they were enlisted for (known to the administrator).
\item The participant must be able to view a possible practical partner.
\item The participant must be able to communicate with the possible practical partner to verify the matching.
\item The participant must be able to accept or decline the practical partner into his practical group.
\item The participant must be able to communicate within the group.
\end{itemize}
\TODO{Add must haves to the peer matching requirements}

\reqr{Should have}
\begin{itemize}
\item The participant should be able to ask a currently existing group if he can join them.
\item The participant should be able to use their Linkedin profile to log in to the application.
\item In case the participant already has an account, the participant should be able to add the course to his profile using the teachers url.
\item The participant should be able to receive an invitation from a group.
\item The participant should be able to accept the invitation from a group.
\end{itemize}

\reqr{Could have}
\begin{itemize}
\item The participant could be able to set his preferences for his possible partner.
\item The participant could be able to search for other participants using information that is available in the profile.
\end{itemize}

\reqr{Would have}
\begin{itemize}
\item \TODO{Add would haves to the peer matching requirements}
\end{itemize}


\subsubsection{Actor 2: MOOC teacher/MOOC administrator}

\reqr{Must have}
\begin{itemize}
\item The administrator must be able to create a MOOC and add this to his account.
\item The administrator must be able to open and close the particpant registration for a MOOC.
\item The administrator must be able to manage the list of participants and groups of participants.
\item The administrator must be able to create an url that he can use to invite the participants.
\end{itemize}

\reqr{Should have}
\begin{itemize}
\item The administrator should be able to grade the group for its practical.
\item The administrator should be able to set a deadline for a practical.
\item The administrator should be able to communicate with every participant.
\item The administrator should be able to communicate with the groups.
\end{itemize}

\reqr{Could have}
\begin{itemize}
\item The administrator could be able to provide for his own criteria for each of his MOOC.
\item The administrator could be able to add different assignments to the application.
\item The administrator could be able to view what groups have finished what assignments.
\item The administrator could be able to grade the different assignments.
\item The administrator could be able to provide feedback for the different assignments.
\end{itemize}

\reqr{Would have}
\begin{itemize}
\item \TODO{Add would haves to the peer matching requirements}
\end{itemize}


\subsubsection{Actor 3: Groups of participants}
This actor consist of two or more MOOC participants who accepted each other to their group.

\reqr{Must have}
\begin{itemize}
\item The groups must be able to communicate with each other using the application.
\end{itemize}

\reqr{Should have}
\begin{itemize}
\item A group should be able to invite other participants to join their group.
\item A group should be able to view a possible group member.
\item A group should be able to invite a possible group member to join their group.
\item A group should be able to accept an request from a participant to join their group.
\end{itemize}

\reqr{Could have}
\begin{itemize}
\item The group could be able to select their preferences for the new group member.
\item The participants could be able to share files to other group members using this application.
\end{itemize}

\reqr{Would have}
\begin{itemize}
\item The participants would be able upload their source code to the application for verification.
\end{itemize}

\subsection{Kind of system}

\reqr{Must have}
\begin{itemize}
\item The application must be easily accessible to both participants and MOOC organizers.
\item The system must provide account management functionality.
\item The system must use a database to store the different data put in by the user.
\end{itemize}

\reqr{Should have}
\begin{itemize}
\item \TODO{Add should haves to the peer matching requirements}
\end{itemize}

\reqr{Could have}
\begin{itemize}
\item The system could be able to be combined with practical systems already used by MOOC organizers.
\end{itemize}

\reqr{Would have}
\begin{itemize}
\item \TODO{Add would haves to the peer matching requirements}
\end{itemize}

\section{Peer matching requirements and literature study}
The peer matching algorithm can be limited and pushed in the correct direction by the requirements, but leaves room for choices to be made.
To make the correct choices a literature study is required.
This section will first list the requirements set by the client and will continue with a discussion about different possibilities to choose from.
It will discuss a range of possible algorithms that can be applied to the problem and what their respective advantages and disadvantages are.
This section will end with a conclusion about the algorithm we will be using.

\subsection{Peer matching}
This subsection will provide a first idea for the peer matching algorithm and what requirements are made for this algorithm.
This algorithm has all of the information of each user as input and calculates values from this information that can be used to compare and group users.
Using these values, the algorithm can calculate if two users would make a good team.
The input is defined by the previous subsection and the output will be defined in this subsection.
\TODO{Finish Peer matching requirements}

\reqr{Must have}
\begin{itemize}
\item The algorithm must be able to use a list of users as input.
\item The algorithm must be able to return one or more suggestions of practical partners to the user given a list of rejected users.
\item The algorithm must be able to return one or more suggestions of possible groups the participant could join.
\end{itemize}

\reqr{Should have}
\begin{itemize}
\item The algorithm uses basic user provided data (e.g. education, preferences) to build up a profile for a MOOC participant.
\item The algorithm can calculate whether two people will be good practical partners using the basic user provided data.
\item The algorithm can return a possible practical partners using the basic user provided data.
\end{itemize}

\reqr{Could have}
\begin{itemize}
\item The algorithm uses advanced user provided data (e.g. asked questions, programming tests) to build up an extended profile for a MOOC participant.
\item The algorithm can calculate whether two people will be good practical partners using the advanced user provided data.
\item The algorithm can return a possible practical partner using the basic user provided data.
\item The algorithm could be able to respond to the preferences of groups.
\item The algorithm could be able to respond to the preferences of a participant.
\end{itemize}

\reqr{Would have}
\begin{itemize}
\item The algorithm would have to be able to use information gathered from rejections to improve the matching.
\end{itemize}

\subsection{Literature study}
\cite{cmu}
\TODO{Add different algorithms and a comparison}

\section{Conclusion}

\newpage
\bibliographystyle{plainnat}
\bibliography{references}

\end{document}