\documentclass[]{article}
\usepackage{a4wide}
\usepackage{xcolor}
\usepackage[numbers]{natbib}
\usepackage[colorlinks,linkcolor=black,urlcolor=blue,citecolor=black]{hyperref}

\newcommand{\TODO}[1]{{\color{red}\textbf{TODO: #1}}}
\renewcommand*\contentsname{Table of Contents}

\title{Peer Matching Research}
\author{Marijn Goedegebure \and
	Floris Verburg \and
	Freek van Tienen}
\date{}

\begin{document}
\maketitle

\begin{abstract}
This document will explain the design process of the Peer Matching system made for the MOOCs, Massive Online Open Courses.
\TODO{Some more text}
\end{abstract}

\newpage

\tableofcontents

\newpage

\section{Summary}
\TODO{Remove test citation\cite{policella07}}

\section{Introduction}
\TODO{Write the introduction}

\section{Problem definition}
More and more people from all over the world with different backgrounds are participating in the MOOCs.
During most of the MOOCs practical assignments need to be made with a partner or in a group.
But finding the correct partner or group members can become really difficult when you don't see each other during courses and haven't even met most of the people from you course.
This is where our Peer Matching system \TODO{Continue problem definition}

\section{The implementations,	the conclusions, discuss and recommendations}

\section{Conclusion}

\newpage
\bibliographystyle{plainnat}
\bibliography{references}

\end{document}