\documentclass[]{article}
\usepackage{a4wide}
\usepackage{xcolor}
\usepackage[numbers]{natbib}
\usepackage[colorlinks,linkcolor=black,urlcolor=blue,citecolor=black]{hyperref}

\newcommand{\TODO}[1]{{\color{red}\textbf{TODO: #1}}}
\renewcommand*\contentsname{Table of Contents}

\title{Peer Matching Research}
\author{Marijn Goedegebure \and
	Floris Verburg \and
	Freek van Tienen}
\date{}

\begin{document}
\maketitle

\begin{abstract}
This document will explain the design process of the Peer Matching system made for the MOOCs, Massive Online Open Courses.
\TODO{Some more text}
\end{abstract}

\newpage

\tableofcontents

\newpage

\section{Summary}
\TODO{Remove test citation\cite{policella07}}

\section{Introduction}
\TODO{Write the introduction}

\section{Problem definition}
More and more people from all over the world with different backgrounds are participating in the MOOCs.
During most of the MOOCs practical assignments need to be made with a partner or in a group.
But finding the correct partner or group members can become really difficult when you don't see each other during courses and haven't even met most of the people from you course.

Most of the time people are searching partners or group members on social media like Facebook or Twitter.
They just try to look through some of the participants names in the course, search for them online trough social media and check if they are a capable partner or group member.
This approach takes a lot of time and most of the time there are too much MOOC participants to go trough all of them.

These approaches often doesn't lead to the "most optimal" solution, but more to a group or set of partners which share the same interest and are able to work together.
Next to the fact that you want a socially stable group or practicum partner you also want to combine as much different experiences as possible.
For example if both practicum partners a good at programming but also both don't like, or aren't good at writing documents and papers.
Then you would be better of by having one person who is good at programming and a person who is better at writing documents.

\TODO{Finish problem definition}


\section{The implementations,	the conclusions, discuss and recommendations}

\section{Requirements}
This section will describe what requirements are set for the product.
These requirements are partly provided by the company and further filled in by our team.
The resulting requirements were completely talked through before the start of the implementation and thus provide a solid base for the rest of our project.
We first define four global requirements who capture the entire project, these requirements are split up in requirements that are measurable and that are assigned to the appropriate MoSCoW method category.

\subsection{Usage user information}
This subsection will provide information about what data is interesting for our peer matching algorithm and how this data is acquired.

\subsection{Peer matching}
This subsection will provide a first idea for the peer matching algorithm and what requirements are made for this algorithm.
The input is defined by the previous subsection and the output will be defined in this subsection.

\subsection{Usage of the system}
The usage of the system is defined by the actions the users can do, these actions are defined in requirements that are listed in this subsection.
The system has different actors that can interact with the system in different ways, so the requirements and actions are categorised by actor.

\subsection{Kind of system}
The previous section limits the possible kinds of systems that are appropriate for our cause.
The requirements that are provided in this section will further limit the kinds of systems.
The MoSCoW categories can be used to compare different kinds of systems and select the best fitting.

\section{Conclusion}

\newpage
\bibliographystyle{plainnat}
\bibliography{references}

\end{document}