\documentclass[]{article}
\usepackage{a4wide}
\usepackage{xcolor}
\usepackage[numbers]{natbib}
\usepackage[colorlinks,linkcolor=blue,urlcolor=blue,citecolor=blue]{hyperref}

\newcommand{\TODO}[1]{{\color{red}\textbf{TODO: #1}}}
\renewcommand*\contentsname{Table of Contents}

\title{Peer Matching Research}
\author{Marijn Goedegebure \and
	Floris Verburg \and
	Freek van Tienen}
\date{}

\begin{document}
\maketitle

\begin{abstract}
This document will explain the design process of the Peer Matching system made for the MOOC', Massive Online Open Courses.
\TODO{Some more text}
\end{abstract}

\tableofcontents

\section{Summary}
\TODO{Remove test citation\cite{policella07}}

\section{Introduction}
\TODO{Write the introduction}

\section{Problem definition and Problem analysis}

\section{The implementations,	the conclusions, discuss and recommendations}

\section{Requirements}
This section will describe what requirements are set for the product.
These requirements are partily provided by the company and further filled in by our team.
The resulting requirements were completely talked through before the start of the implementation and thus provide a solid base for the rest of our project.
We first define four global requirements who capture the entire project, these requirements are split up in requirements that are measureable and that are assigned to the appropiate MoSCoW method category.

\subsection{Usage user information}

\subsection{Peer matching}

\subsection{Usage of the system}

\subsection{Kind of system}

\section{Conclusion}

\newpage
\bibliographystyle{plainnat}
\bibliography{references}

\end{document}