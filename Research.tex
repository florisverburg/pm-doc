\documentclass[]{article}
\usepackage{a4wide}
\usepackage{xcolor}
\usepackage[numbers]{natbib}
\usepackage[colorlinks,linkcolor=black,urlcolor=blue,citecolor=black]{hyperref}

\newcommand{\TODO}[1]{{\color{red}\textbf{TODO: #1}}}
\renewcommand*\contentsname{Table of Contents}

\title{Peer Matching Research}
\author{Marijn Goedegebure \and
	Floris Verburg \and
	Freek van Tienen}
\date{}

\begin{document}
\maketitle

\begin{abstract}
This document will explain the design process of the Peer Matching system made for the MOOCs, Massive Online Open Courses.
\TODO{Some more text}
\end{abstract}

\newpage

\tableofcontents

\newpage

\section{Summary}
\TODO{Remove test citation\cite{loughry2010design}}

\section{Introduction}
\TODO{Write the introduction}

\section{Problem definition}
More and more people from all over the world with different backgrounds are participating in the MOOCs.
During most of the MOOCs practical assignments need to be made with a partner or in a group.
But finding the correct partner or group members can become really difficult when you don't see each other during courses and haven't even met most of the people from you course.

Most of the time people are searching partners or group members on social media like Facebook or Twitter.
They just try to look through some of the participants names in the course, search for them online trough social media and check if they are a capable partner or group member.
This approach takes a lot of time and most of the time there are too much MOOC participants to go trough all of them.

These approaches often doesn't lead to the "most optimal" solution, but more to a group or set of partners which share the same interest and are able to work together.
Next to the fact that you want a socially stable group or practicum partner you also want to combine as much different experiences as possible.
For example if both practicum partners a good at programming but also both don't like, or aren't good at writing documents and papers.
Then you would be better of by having one person who is good at programming and a person who is better at writing documents.

\TODO{Finish problem definition}

\section{Requirements}
This section will describe what requirements are set for the product.
These requirements are partly provided by the company and further filled in by our team.
The resulting requirements were completely talked through before the start of the implementation and thus provide a solid base for the rest of our project.
We first define four global requirements who capture the entire project, these requirements are split up in requirements that are measurable and that are assigned to the appropriate MoSCoW\cite{highsmith2001agile} method category.

\subsection{Usage user information}
This subsection will provide information about what data is interesting for our peer matching algorithm and how this data is acquired.
Some data is necessary to collect. 
Other data would be nice to have to give an better result, but is not essential for basic matching. 
We distinguish the essential data and the not essential data.\cite{cmu}

\emph{Must have:}
\begin{itemize}
\item The algorithm must know the size of the groups that must be created
\item Users must declare their prior knowledge.
\item Users must declare their previous experiences.
\item Users must declare their skills.
\end{itemize}

\emph{Should have:}
\begin{itemize}
\item Users should declare their motivation. 
Mixing students with different motivations within a group can cause tensions and problems.
\item Users should declare their personality. 
\end{itemize}

\emph{Could have:}
\begin{itemize}
\item The users could indicate which role they can perform best, for example project manger, data analyst, writer. 
\end{itemize}

\emph{Would have:}
\begin{itemize}
\item \TODO{Add would haves to the peer matching requirements}
\end{itemize}

\subsection{Peer matching}
This subsection will provide a first idea for the peer matching algorithm and what requirements are made for this algorithm.
The input is defined by the previous subsection and the output will be defined in this subsection.
\TODO{Finish Peer matching requirements}
\emph{Must have:}
\begin{itemize}
\item The algorithm must return possible practical partners to the user.
\item The algorithm must be able to return more then one possible practical partner.
\item The algorithm must be able to respond to a rejection by the user.
\end{itemize}

\emph{Should have:}
\begin{itemize}
\item The algorithm uses basic user provided data (e.g. education, preferences) to build up a profile for a MOOC participant.
\item The algorithm can calculate whether two people will be good practical partners using the basic user provided data.
\item The algorithm can return a possible practical partners using the basic user provided data.
\end{itemize}

\emph{Could have:}
\begin{itemize}
\item The algorithm uses advanced user provided data (e.g. asked questions, programming tests) to build up an extended profile for a MOOC participant.
\item The algorithm can calculate whether two people will be good practical partners using the advanced user provided data.
\item The algorithm can return a possible practical partner using the basic user provided data.
\end{itemize}

\emph{Would have:}
\begin{itemize}
\item \TODO{Add would haves to the peer matching requirements}
\end{itemize}

\subsection{Usage of the system}
The usage of the system is defined by the actions the users can do, these actions are defined in requirements that are listed in this subsection.
The system has different actors that can interact with the system in different ways, so the requirements and actions are categorised by actor.

\subsubsection{Actor 1: MOOC participant}
\emph{Must have:}
\begin{itemize}
\item The participant must be able to view the different MOOCs without registering or logging in.
\item The participant must be able to inform him/herself about the usage of the application without registering or logging in.
\item The participant must be able to register to the application.
\item The participant must be able to log in to the application.
\item The participant must be able to register to the MOOCs which they were enlisted for (known to the administrator).
\item The participant must be able to view a possible practical partner.
\item The participant must be able to communicate with the possible practical partner to verify the matching.
\item The participant must be able to accept or decline the practical partner into his practical group.
\item The participant must be able to communicate within the group.
\end{itemize}
\TODO{Add must haves to the peer matching requirements}

\emph{Should have:}
\begin{itemize}
\item \TODO{Add should haves to the peer matching requirements}
\end{itemize}

\emph{Could have:}
\begin{itemize}
\item \TODO{Add could haves to the peer matching requirements}
\end{itemize}

\emph{Would have:}
\begin{itemize}
\item \TODO{Add would haves to the peer matching requirements}
\end{itemize}

\subsubsection{Actor 2: MOOC teacher/MOOC administrator}
\emph{Must have:}
\begin{itemize}
\item The administrator must be able to create a MOOC and add this to his account.
\item The administrator must be able to invite participants to join the MOOC.
\item The administrator must be able to manage the list of participants and groups of participants.
\end{itemize}

\emph{Should have:}
\begin{itemize}
\item The administrator should be able to grade the group for its practical.
\item The administrator should be able to set deadlines for each practical.
\item The administrator should be able to easily communicate with every participant.
\item The administrator should be able to communicate with the groups.
\end{itemize}

\emph{Could have:}
\begin{itemize}
\item The administrator could be able to provide for his own criteria for each of his MOOC.
\item The administrator could be able to add different assignments to the application.
\item The administrator could be able to grade the different assignments.
\item The administrator could be able to provide feedback for the different assignments.
\end{itemize}

\emph{Would have:}
\begin{itemize}
\item \TODO{Add would haves to the peer matching requirements}
\end{itemize}

\subsubsection{Actor 3: Groups of participants}
This actor consist of two or more MOOC participants who accepted each other to their group.
\emph{Must have:}
\begin{itemize}
\item The groups must be able to communicate with each other using the application.
\end{itemize}

\emph{Should have:}
\begin{itemize}
\item \TODO{Add should haves to the peer matching requirements}
\end{itemize}

\emph{Could have:}
\begin{itemize}
\item \TODO{Add should haves to the peer matching requirements}
\end{itemize}

\emph{Would have:}
\begin{itemize}
\item \TODO{Add would haves to the peer matching requirements}
\end{itemize}

\subsection{Kind of system}
The previous section limits the possible kinds of systems that are appropriate for our cause.
The requirements that are provided in this section will further limit the kinds of systems.
The MoSCoW categories can be used to compare different kinds of systems and select the best fitting.

\emph{Must have:}
\begin{itemize}
\item \TODO{Add must haves to the peer matching requirements}
\end{itemize}

\emph{Should have:}
\begin{itemize}
\item \TODO{Add should haves to the peer matching requirements}
\end{itemize}

\emph{Could have:}
\begin{itemize}
\item \TODO{Add could haves to the peer matching requirements}
\end{itemize}

\emph{Would have:}
\begin{itemize}
\item \TODO{Add would haves to the peer matching requirements}
\end{itemize}

\section{Chosen implementations for requirements}
Some requirements limit the possible choices for a part of the system, but still leaves some questions unanswered. This section will provide an answer to these questions.\\
\TODO{Continue working on this section}\\
\TODO{Add requirements that leave room for different implementations (e.g. peer matching algorithm, what method is used)}

\section{Conclusion}

\newpage
\bibliographystyle{plainnat}
\bibliography{references}

\end{document}