\documentclass[]{article}
\usepackage{a4wide}
\usepackage{xcolor}
\usepackage[numbers]{natbib}
\usepackage[colorlinks,linkcolor=black,urlcolor=blue,citecolor=black]{hyperref}

\newcommand{\TODO}[1]{{\color{red}\textbf{TODO: #1}}}
\newcommand{\reqr}[1]{{\noindent\emph{#1:}}}
\renewcommand*\contentsname{Table of Contents}

\title{Plan of Action}
\author{Marijn Goedegebure \and
	Floris Verburg \and
	Freek van Tienen}
\date{}

\begin{document}
\maketitle

\begin{abstract}
\TODO{Summarize the most important conclusions of this document, add the demands for the validity of this document}
\end{abstract}

\newpage

\tableofcontents

\newpage
\TODO{Fill each section and subsection with the appropriate information provided in the standard plan of action document}
\section{Preface}
This document is the Plan of Action report that has been written for the Peer Matching assignment for the Bachelor Project course.
This course is the last project based course of the Bachelor of Science at the TU Delft in the Netherlands.
The assignment was issued by Dr. L.J.M. Rothkrantz of the TU Delft.
In this document, we will outline the plan of action for this project.
In the first section we will introduce the assignment and explain why it was issued.
In the second chapter we will describe the current situation, the goal of the project, the project assignment and the deliverables.
In the third chapter we will define our approach and planning for the time that is available to us.
In the fourth section we will define the project approach, who will be working on the project, who is in charge of what and administrative procedures.
We will also be describing how we finance the  resource that we require, how we will report the progress and what technical resources we will be using.
In the final section we will describe our approach to assuring quality of the final product.
\section{Introduction}
\subsection{Background and motivation of the assignment}
\subsection{Agreement and adjustment of the plan of action}

\section{Assignment description}
In this section we will describe the definition of our Peer Matching assignment.

\subsection{Client}
The assignment was written by Dr. L.J.M. Rothkrantz of the TU Delft and is our main client.
Dr. L.J.M. Rothkrantz asked Dragos Datcu from the TU Delft to advise us on our design and implementation of the Peer Matching assignment.

The target users of the system are MOOC participants and MOOC teachers, specifically for the Computer Science courses. The system is easily extendable to provide more different MOOCs then Computer Science courses, but will not be supported by default.

\subsection{Contact information}
\emph{Team}

Marijn Goedegebure

Floris Verburg

Freek van Tienen

\noindent\emph{Project supervisors}

Dr. L.J.M. Rothkrantz

Dragos Datcu

\subsection{Problem description}
MOOCs around the world are growing in size and thus the practical assignments need to be easy to manage.
The growth also means that it is more difficult for each participant to find a possible practical partner due to the large amount of people.
It also becomes really difficult to find the correct partner or group members when you don't see each other during courses and haven't even met most of the people from your course.

Most of the time participants of MOOCs are searching group members on social media like Facebook or Twitter.
They just try to look through some of the participants names in the course, search for them on-line through social media and check if they are a capable partner or group member.
This approach takes a lot of time and most of the time there are too much MOOC participants to go trough all of them.

Next to the fact that this process is very time consuming, this process also often leads to a sub optimal solution.
This is due to the problem that people often look for partners of group members which share the same interests.
While this often makes it easy to work together, this will not guarantee that you have a good functioning group and enough knowledge inside the group.

We would like to improve upon this process by creating an application that can be used to create practical groups without the participants being required to go search intensively trough all the particpants.
Participants should be able to easily access the application, fill in some relevant information about themselves and then be recommended possible practical partners or group members.

The interaction with the system is a pretty straight forward application, but the recommendation of possible group members is more difficult.
One of the problems we are facing is the question: What user information is useful to determine whether two people can be good practical partners?
This and other questions will be answered in this document.

\subsection{Goal}
\subsection{Assignment formulation}
\subsection{Deliverables}
\subsection{Preconditions}
\subsection{Risks}
\section{Approach and time schedule}
\subsection{Introduction}
\subsection{Methods and techniques}
\subsection{Proceedings}
\subsection{Planning}
\section{Project approach}
\subsection{Introduction}
\subsection{Stakeholders}
\subsection{Information}
\subsection{Facilities}
\section{Quality assurance}
\subsection{Introduction}
\subsection{Quality}



\end{document}