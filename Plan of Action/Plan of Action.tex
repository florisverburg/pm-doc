\documentclass[]{article}
\usepackage{a4wide}
\usepackage{xcolor}
\usepackage[numbers]{natbib}
\usepackage[colorlinks,linkcolor=black,urlcolor=blue,citecolor=black]{hyperref}

\newcommand{\TODO}[1]{{\color{red}\textbf{TODO: #1}}}
\newcommand{\reqr}[1]{{\noindent\emph{#1:}}}
\renewcommand*\contentsname{Table of Contents}

\title{Plan of Action}
\author{Marijn Goedegebure \and
	Floris Verburg \and
	Freek van Tienen}
\date{}

\begin{document}
\maketitle

\begin{abstract}
\TODO{Summarize the most important conclusions of this document, add the demands for the validity of this document}
\end{abstract}

\newpage

\tableofcontents

\newpage
\TODO{Fill each section and subsection with the appropriate information provided in the standard plan of action document}
\section{Preface}
This document is the Plan of Action report that has been written for the Peer Matching assignment for the Bachelor Project course.
This course is the last project based course of the Bachelor of Science at the TU Delft in the Netherlands.
The assignment was issued by Dr. L.J.M. Rothkrantz of the TU Delft.
In this document, we will outline the plan of action for this project.
In the first section we will introduce the assignment and explain why it was issued.
In the second chapter we will describe the current situation, the goal of the project, the project assignment and the deliverables.
In the third chapter we will define our approach and planning for the time that is available to us.
In the fourth section we will define the project approach, who will be working on the project, who is in charge of what and administrative procedures.
We will also be describing how we finance the  resource that we require, how we will report the progress and what technical resources we will be using.
In the final section we will describe our approach to assuring quality of the final product.
\section{Introduction}
In this section we will provide an introduction to the assignment.
We will start by giving a company outline followed by giving a background and motivation of the assignment.
This will answer the question: why has this assignment been issued? 
\subsection{Company outline}
The assignment is issued by Dr. L.J.M. Rothkrantz who is part of the Interactive Intelligence group in the Intelligent Systems department.
This department is located in the EEMCS faculty at the TU Delft.
The research mission of the Computer Science department is as follows:
"The research mission of Computer Science within the Faculty of EEMCS is to contribute to the advancement of science, engineering, and design in the broad fields of autonomous distributed systems and information analysis and interaction."\cite{cstudelftnl}

\subsection{Background and motivation of the assignment}
Each year many students go to an university to follow the predefined bachelor and/or master programs, but not everyone has the time and intent to follow an entire program.
There is a demand from all sorts of people to be able to follow a course while their main occupation isn't that of a student.
For example: a working parent could be very interested in such a course.
That is why universities introduced MOOCs.

The basic idea of MOOCs is that students remote in place and time follow the lectures, can participate in a practical assignment and can make an exam.
If they score sufficient, they are rewarded with a certificate that states their succes in the course.

The concept of the MOOCs is a very new concept that still has many areas to improve upon.
One of these areas is the area of practical assignments.
Currently, for a Computer Science MOOC at the TU Delft, multiple practical assignments are required.
The participants are required to form groups to make these assignments.
After the announcement the participants either use mail, Facebook or  Twitter to contact other people in order to form the groups.

We would like to improve upon this concept and try to provide a more fitting approach.
\section{Assignment description}
In this section we will describe the definition of our Peer Matching assignment.

\subsection{Client}
The assignment was written by Dr. L.J.M. Rothkrantz of the TU Delft and is our main client.
Dr. L.J.M. Rothkrantz asked Dragos Datcu from the TU Delft to advise us on our design and implementation of the Peer Matching assignment.

The target users of the system are MOOC participants and MOOC teachers, specifically for the Computer Science courses. The system is easily extendable to provide more different MOOCs then Computer Science courses, but will not be supported by default.

\subsection{Contact information}
\emph{Team}

Marijn Goedegebure

Floris Verburg

Freek van Tienen

\noindent\emph{Project supervisors}

Dr. L.J.M. Rothkrantz

Dragos Datcu

\subsection{Problem description}
MOOCs around the world are growing in size and thus the practical assignments need to be easy to manage.
The growth also means that it is more difficult for each participant to find a possible practical partner due to the large amount of people.
It also becomes really difficult to find the correct partner or group members when you don't see each other during courses and haven't even met most of the people from your course.

Most of the time participants of MOOCs are searching group members on social media like Facebook or Twitter.
They just try to look through some of the participants names in the course, search for them on-line through social media and check if they are a capable partner or group member.
This approach takes a lot of time and most of the time there are too much MOOC participants to go trough all of them.

Next to the fact that this process is very time consuming, this process also often leads to a sub optimal solution.
This is due to the problem that people often look for partners of group members which share the same interests.
While this often makes it easy to work together, this will not guarantee that you have a good functioning group and enough knowledge inside the group.

We would like to improve upon this process by creating an application that can be used to create practical groups without the participants being required to go search intensively trough all the participants.
Participants should be able to easily access the application, fill in some relevant information about themselves and then be recommended possible practical partners or group members.

The interaction with the system is a pretty straight forward application, but the recommendation of possible group members is more difficult.
One of the problems we are facing is the question: What user information is useful to determine whether two people can be good practical partners?
This and other questions will be answered in this document.

\subsection{Goal}
To provide an overview of what the software must be capable of, we define four global requirements for the system.
By defining these four global requirements we provide a foundation of what the system must be able to do.
\begin{itemize}
\item MOOC teachers must be able to create, manage and edit courses for which participants of their MOOC are able to register.

\item MOOC students must be able to register and deregister for a MOOC practical assignment.

\item The system must be capable to use personal data provided by the user to find a suggestion of an appropriate practical partner or group member.
It must be possible to adjust the definition (parameters) of the appropriate practical partner or group member, and search trough several suggestions.
\end{itemize}

\subsection{Assignment formulation}
\subsection{Deliverables}
\subsection{Preconditions}
\subsection{Risks}
\section{Approach and time schedule}
\TODO{Gaat Marijn doen}\\
In this section we will be describing the methods, techniques and tools that we will be using to develop our application. We will also be giving a global overview of the planning.

\subsection{Methods and techniques}
We will be developing our application in the programming language Java. We will combine this with the Play framework to provide us with a solid basis with which we can develop our web application.

We will being using the following applications and technologies:
\begin{itemize}
\item IntelliJ IDEA, is our Integrated Development Environment (IDE) which will facilitate the development and testing of our code.
IntelliJ will be integrated with the Play framework which will be helpful during development.
\item GitHub, is our on-line code and documentation repository.
It provides for a version control system, an issue tracker and code review possibilities.
We will be using it to store all our code and documentation's code (LaTeX).
\item Cloudbees
Cloudbees will be used to run our test environment, our continuous integration and the release environment.
It uses Jenkins for the continuous integration.
\item Play framework
The play framework gives us the possibility to easily create a web application using Java.
\item Findbugs is a plug in for Cloudbees and IntelliJ that gives us the possibility to let our java code be checked for small bugs using static analysis.
\item JaCoCo is a plug in for Cloudbees and IntelliJ that provides us with data analysis about our code coverage.
\item Checkstyle is a plug in for Cloudbees and IntelliJ that checks the code for coding standards.
This makes it ideal to enforce the coding standard for our project.
\end{itemize}
\subsection{Proceedings}
\subsection{Planning}
\section{Project approach}
\subsection{Introduction}
\subsection{Stakeholders}
\subsection{Information}
\subsection{Facilities}
\section{Quality assurance}
\subsection{Introduction}
\subsection{Quality}

\newpage
\bibliographystyle{plainnat}
\bibliography{references}

\end{document}